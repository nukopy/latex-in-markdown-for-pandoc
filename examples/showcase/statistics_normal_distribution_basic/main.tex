% Options for packages loaded elsewhere
\PassOptionsToPackage{unicode}{hyperref}
\PassOptionsToPackage{hyphens}{url}
\PassOptionsToPackage{space}{xeCJK}
\documentclass[
  11pt,
  a4paper,
]{extarticle}
\usepackage{xcolor}
\usepackage[top=10mm,bottom=15mm,left=22mm,right=22mm]{geometry}
\usepackage{amsmath,amssymb}
\setcounter{secnumdepth}{-\maxdimen} % remove section numbering
\usepackage{iftex}
\ifPDFTeX
  \usepackage[T1]{fontenc}
  \usepackage[utf8]{inputenc}
  \usepackage{textcomp} % provide euro and other symbols
\else % if luatex or xetex
  \usepackage{unicode-math} % this also loads fontspec
  \defaultfontfeatures{Scale=MatchLowercase}
  \defaultfontfeatures[\rmfamily]{Ligatures=TeX,Scale=1}
\fi
\usepackage{lmodern}
\ifPDFTeX\else
  % xetex/luatex font selection
  \setmainfont[]{Hack Nerd Font Mono}
  \setsansfont[]{Hack Nerd Font Mono}
  \setmonofont[]{Hack Nerd Font Mono}
  \ifXeTeX
    \usepackage{xeCJK}
    \setCJKmainfont[]{Hiragino Sans W3}
  \fi
  \ifLuaTeX
    \usepackage[]{luatexja-fontspec}
    \setmainjfont[]{Hiragino Sans W3}
  \fi
\fi
% Use upquote if available, for straight quotes in verbatim environments
\IfFileExists{upquote.sty}{\usepackage{upquote}}{}
\IfFileExists{microtype.sty}{% use microtype if available
  \usepackage[]{microtype}
  \UseMicrotypeSet[protrusion]{basicmath} % disable protrusion for tt fonts
}{}
\makeatletter
\@ifundefined{KOMAClassName}{% if non-KOMA class
  \IfFileExists{parskip.sty}{%
    \usepackage{parskip}
  }{% else
    \setlength{\parindent}{0pt}
    \setlength{\parskip}{6pt plus 2pt minus 1pt}}
}{% if KOMA class
  \KOMAoptions{parskip=half}}
\makeatother
\pagestyle{plain}
\setlength{\emergencystretch}{3em} % prevent overfull lines
\providecommand{\tightlist}{%
  \setlength{\itemsep}{0pt}\setlength{\parskip}{0pt}}
\usepackage{amssymb}
\usepackage{mathtools}
\usepackage{bm}
\usepackage{setspace}
\setlength{\parskip}{0.75em}
\setstretch{1.25}
\newcommand{\E}{\mathbb{E}}
\newcommand{\Var}{\mathrm{Var}}
\newcommand{\diff}{\mathrm{d}}
\numberwithin{equation}{section}
\usepackage{amsmath}
\usepackage{amsthm}
\theoremstyle{plain}
\newtheorem{thm}{定理}
\newtheorem*{thm*}{定理}
\theoremstyle{definition}
\newtheorem{dfn}{定義}
\usepackage{bookmark}
\IfFileExists{xurl.sty}{\usepackage{xurl}}{} % add URL line breaks if available
\urlstyle{same}
\hypersetup{
  pdftitle={正規分布の期待値(平均)、分散、標準偏差とその導出},
  pdfauthor={統計太郎},
  hidelinks,
  pdfcreator={LaTeX via pandoc}}

\title{正規分布の期待値(平均)、分散、標準偏差とその導出}
\author{統計太郎}
\date{2025-10-06}

\begin{document}
\maketitle

\subsection{参考}\label{ux53c2ux8003}

本ドキュメントは以下の資料を参考に作成しています。

\begin{itemize}
\tightlist
\item
  Web サイト:
  \href{https://mathlandscape.com/normal-distrib-ev/}{数学の景色 -
  正規分布の期待値(平均)・分散・標準偏差とその導出証明}
\item
  Web サイト: \href{https://mathlandscape.com/latex-amsthm/}{数学の景色
  - 【LaTeX】定理環境 amsthm パッケージの使い方を徹底解説}
\end{itemize}

\tableofcontents

\newpage

\subsection{イントロダクション:正規分布の期待値、分散、標準偏差}\label{ux30a4ux30f3ux30c8ux30edux30c0ux30afux30b7ux30e7ux30f3ux6b63ux898fux5206ux5e03ux306eux671fux5f85ux5024ux5206ux6563ux6a19ux6e96ux504fux5dee}

平均 \mu \in \mathbb{R}、分散 \sigma\^{}2 \textgreater{} 0
をもつ正規分布 X \sim N(\mu, \sigma\^{}2) の確率密度関数は、

p(x) = \frac{1}{\sqrt{2\pi \sigma^2}}
\exp\left(-\frac{(x-\mu)^2}{2\sigma^2}\right)

となります。このとき、期待値、分散、標準偏差はそれぞれ次のとおりとなります。

\mathbb{E}[X] = \mu, \qquad \mathrm{Var}(X) = \sigma\^{}2,
\qquad \sqrt{\mathrm{Var}(X)} = \sigma.

以下では、正規分布の定義から期待値、分散、標準偏差を導出する方法を紹介します。

\subsection{正規分布の定義\#\#
正規分布の定義}\label{ux6b63ux898fux5206ux5e03ux306eux5b9aux7fa9-ux6b63ux898fux5206ux5e03ux306eux5b9aux7fa9}

まず、正規分布の確率密度関数の定義を確認します。

\$\$

\begin{dfn}

X を確率変数、\mu \in \mathbb{R}、\sigma > 0 とする。

平均 \mu \in \mathbb{R}、分散 \sigma^2 > 0 をもつ正規分布 X \sim N(\mu, \sigma^2) の確率密度関数は

p(x) = \frac{1}{\sqrt{2\pi \sigma^2}} \exp\left(-\frac{(x-\mu)^2}{2\sigma^2}\right)

\end{dfn}

\$\$

\subsection{証明}\label{ux8a3cux660e}

\subsection{正規分布の分散の導出:
定義から直接証明する方法}\label{ux6b63ux898fux5206ux5e03ux306eux5206ux6563ux306eux5c0eux51fa-ux5b9aux7fa9ux304bux3089ux76f4ux63a5ux8a3cux660eux3059ux308bux65b9ux6cd5}

\subsection{正規分布の標準偏差について}\label{ux6b63ux898fux5206ux5e03ux306eux6a19ux6e96ux504fux5deeux306bux3064ux3044ux3066}

\end{document}
