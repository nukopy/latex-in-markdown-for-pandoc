% Options for packages loaded elsewhere
\PassOptionsToPackage{unicode}{hyperref}
\PassOptionsToPackage{hyphens}{url}
\PassOptionsToPackage{space}{xeCJK}
\documentclass[
  11pt,
  a4paper,
]{extarticle}
\usepackage{xcolor}
\usepackage[top=10mm,bottom=15mm,left=22mm,right=22mm]{geometry}
\usepackage{amsmath,amssymb}
\setcounter{secnumdepth}{-\maxdimen} % remove section numbering
\usepackage{iftex}
\ifPDFTeX
  \usepackage[T1]{fontenc}
  \usepackage[utf8]{inputenc}
  \usepackage{textcomp} % provide euro and other symbols
\else % if luatex or xetex
  \usepackage{unicode-math} % this also loads fontspec
  \defaultfontfeatures{Scale=MatchLowercase}
  \defaultfontfeatures[\rmfamily]{Ligatures=TeX,Scale=1}
\fi
\usepackage{lmodern}
\ifPDFTeX\else
  % xetex/luatex font selection
  \ifXeTeX
    \usepackage{xeCJK}
    \setCJKmainfont[]{Hiragino Sans W3}
  \fi
  \ifLuaTeX
    \usepackage[]{luatexja-fontspec}
    \setmainjfont[]{Hiragino Sans W3}
  \fi
\fi
% Use upquote if available, for straight quotes in verbatim environments
\IfFileExists{upquote.sty}{\usepackage{upquote}}{}
\IfFileExists{microtype.sty}{% use microtype if available
  \usepackage[]{microtype}
  \UseMicrotypeSet[protrusion]{basicmath} % disable protrusion for tt fonts
}{}
\makeatletter
\@ifundefined{KOMAClassName}{% if non-KOMA class
  \IfFileExists{parskip.sty}{%
    \usepackage{parskip}
  }{% else
    \setlength{\parindent}{0pt}
    \setlength{\parskip}{6pt plus 2pt minus 1pt}}
}{% if KOMA class
  \KOMAoptions{parskip=half}}
\makeatother
\pagestyle{plain}
\setlength{\emergencystretch}{3em} % prevent overfull lines
\providecommand{\tightlist}{%
  \setlength{\itemsep}{0pt}\setlength{\parskip}{0pt}}
\usepackage{amssymb}
\usepackage{mathtools}
\usepackage{bm}
\usepackage{setspace}
\setlength{\parskip}{0.75em}
\setstretch{1.25}
\newcommand{\E}{\mathbb{E}}
\newcommand{\Var}{\mathrm{Var}}
\newcommand{\diff}{\mathrm{d}}
\numberwithin{equation}{section}
\usepackage{amsmath}
\usepackage{amsthm}
\theoremstyle{plain}
\newtheorem{thm}{定理}
\newtheorem*{thm*}{定理}
\theoremstyle{definition}
\newtheorem{dfn}{定義}
\newtheorem*{dfn*}{定義}
\renewcommand{\proofname}{証明}
\usepackage{bookmark}
\IfFileExists{xurl.sty}{\usepackage{xurl}}{} % add URL line breaks if available
\urlstyle{same}
\hypersetup{
  pdftitle={正規分布の期待値(平均)、分散、標準偏差とその導出},
  pdfauthor={統計太郎},
  hidelinks,
  pdfcreator={LaTeX via pandoc}}

\title{正規分布の期待値(平均)、分散、標準偏差とその導出}
\author{統計太郎}
\date{2025-10-06}

\begin{document}
\maketitle

\maketitle

\tableofcontents

\subsection{参考}\label{ux53c2ux8003}

本ドキュメントは以下の資料を参考に作成している。

\begin{itemize}
\tightlist
\item
  Web サイト:
  \href{https://mathlandscape.com/normal-distrib-ev/}{数学の景色 -
  正規分布の期待値(平均)・分散・標準偏差とその導出証明}
\item
  Web サイト: \href{https://mathlandscape.com/latex-amsthm/}{数学の景色
  - 【LaTeX】定理環境 amsthm パッケージの使い方を徹底解説}
\end{itemize}

\newpage

\subsection{イントロダクション:正規分布の期待値、分散、標準偏差}\label{ux30a4ux30f3ux30c8ux30edux30c0ux30afux30b7ux30e7ux30f3ux6b63ux898fux5206ux5e03ux306eux671fux5f85ux5024ux5206ux6563ux6a19ux6e96ux504fux5dee}

本ドキュメントでは、正規分布の定義から期待値、分散、標準偏差を導出する。

まずは正規分布の期待値、分散、標準偏差の定理を確認する。

\begin{thm*}

確率変数 $X$ が正規分布 $N(\mu, \sigma^2)$ に従うとき($X \sim N(\mu, \sigma^2)$)、$X$ の期待値、分散、標準偏差はそれぞれ次のとおりとなる。

$$
E[X] = \mu, \qquad V(X) = \sigma^2, \qquad \sqrt{V(X)} = \sigma.
$$

\end{thm*}

\subsection{正規分布の定義}\label{ux6b63ux898fux5206ux5e03ux306eux5b9aux7fa9}

まず、正規分布の確率密度関数の定義を確認する。

\begin{dfn*}

$X$ を確率変数、$\mu \in \mathbb{R}$、$\sigma > 0$ とする。$X$ の確率密度関数が

$$
p(x) = \frac{1}{\sqrt{2\pi \sigma^2}} \exp\left(-\frac{(x-\mu)^2}{2\sigma^2}\right)
$$

となるとき、$X$ は正規分布 $N(\mu, \sigma^2)$(normal distribution)に従うといい、$X \sim N(\mu, \sigma^2)$ と表す。

\end{dfn*}

\newpage

\subsection{正規分布の期待値、分散、標準偏差の導出}\label{ux6b63ux898fux5206ux5e03ux306eux671fux5f85ux5024ux5206ux6563ux6a19ux6e96ux504fux5deeux306eux5c0eux51fa}

\subsubsection{正規分布の期待値(平均)の導出}\label{ux6b63ux898fux5206ux5e03ux306eux671fux5f85ux5024ux5e73ux5747ux306eux5c0eux51fa}

\begin{thm*}

$X \sim N(\mu, \sigma^2)$ に対して期待値は $E[X] = \mu$ である。

\end{thm*}

\begin{proof}
正規分布の定義より、確率密度関数は

$$
p(x) = \frac{1}{\sqrt{2\pi \sigma^2}} \exp\left(-\frac{(x-\mu)^2}{2\sigma^2}\right)
$$

と表される。期待値は定義から

\begin{align*}
E[X]
&= \int_{-\infty}^{\infty} x \, p(x) \, \mathrm{d}x \\
&= \int_{-\infty}^{\infty} \bigl((x-\mu) + \mu\bigr) p(x) \, \mathrm{d}x \\
&= \mu \int_{-\infty}^{\infty} p(x) \, \mathrm{d}x + \int_{-\infty}^{\infty} (x-\mu) p(x) \, \mathrm{d}x.
\end{align*}

正規分布は確率密度関数の積分が 1 であるため、第一項は $\mu$ となる。第二項は変数変換 $y = \frac{x-\mu}{\sigma}$ を用いると

\begin{align*}
\int_{-\infty}^{\infty} (x-\mu) p(x) \, \mathrm{d}x
&= \frac{1}{\sqrt{2\pi}} \int_{-\infty}^{\infty} \sigma y \exp\left(-\frac{y^2}{2}\right) \mathrm{d}y \\
&= \frac{\sigma}{\sqrt{2\pi}} \int_{-\infty}^{\infty} y \exp\left(-\frac{y^2}{2}\right) \mathrm{d}y.
\end{align*}

被積分関数 $y \exp(-y^2/2)$ は奇関数であり、積分区間 $(-\infty, \infty)$ は対称であるため、この積分は 0 となる。したがって

$$
E[X] = \mu + 0 = \mu
$$

が得られる。
\end{proof}

\newpage

\subsubsection{正規分布の分散の導出}\label{ux6b63ux898fux5206ux5e03ux306eux5206ux6563ux306eux5c0eux51fa}

\begin{thm*}

$X \sim N(\mu, \sigma^2)$ に対して分散は $V(X) = \sigma^2$ である。

\end{thm*}

\begin{proof}
分散の定義から

$$
V(X) = E\bigl[(X-\mu)^2\bigr] = \int_{-\infty}^{\infty} (x-\mu)^2 p(x) \, \mathrm{d}x
$$

となる。期待値の導出と同様に $y = \frac{x-\mu}{\sigma}$ と置くと

\begin{align*}
V(X)
&= \frac{1}{\sqrt{2\pi}} \int_{-\infty}^{\infty} (\sigma y)^2 \exp\left(-\frac{y^2}{2}\right) \mathrm{d}y \\
&= \frac{\sigma^2}{\sqrt{2\pi}} \int_{-\infty}^{\infty} y^2 \exp\left(-\frac{y^2}{2}\right) \mathrm{d}y.
\end{align*}

ここで積分を評価するために、偶関数性を利用して積分区間を $[0, \infty)$ へ制限し、部分積分を適用する。

\begin{align*}
\int_{-\infty}^{\infty} y^2 \exp\left(-\frac{y^2}{2}\right) \mathrm{d}y
&= 2 \int_0^{\infty} y^2 \exp\left(-\frac{y^2}{2}\right) \mathrm{d}y \\
&= 2 \left[ -y \exp\left(-\frac{y^2}{2}\right) \right]_0^{\infty} + 2 \int_0^{\infty} \exp\left(-\frac{y^2}{2}\right) \mathrm{d}y.
\end{align*}

境界項は 0 に収束するため消える。また、ガウス積分(正規分布の正規化条件)\footnote{\href{https://mathlandscape.com/gauss-integral/}{Web サイト: 数学の景色 - ガウス積分のさまざまな形とその証明 5 つ}}より

$$
\int_{-\infty}^{\infty} \exp\left(-\frac{y^2}{2}\right) \mathrm{d}y = \sqrt{2\pi}
$$

が成り立つので、半区間の積分はその半分であり、$\int_0^{\infty} \exp(-y^2/2) \mathrm{d}y = \sqrt{2\pi}/2$ となる。以上から

$$
\int_{-\infty}^{\infty} y^2 \exp\left(-\frac{y^2}{2}\right) \mathrm{d}y = 2 \cdot \frac{\sqrt{2\pi}}{2} = \sqrt{2\pi}
$$

が得られる。したがって

$$
V(X) = \frac{\sigma^2}{\sqrt{2\pi}} \times \sqrt{2\pi} = \sigma^2
$$

が示された。
\end{proof}

\newpage

\subsubsection{正規分布の標準偏差の導出}\label{ux6b63ux898fux5206ux5e03ux306eux6a19ux6e96ux504fux5deeux306eux5c0eux51fa}

標準偏差の定義は、分散の平方根 \(\sqrt{V(X)}\) である。分散が
\(V(X) = \sigma^2\) と導出されたことから、

\[
\sqrt{V(X)} = \sqrt{\sigma^2} = \sigma
\]

が直ちに得られる。

\end{document}
